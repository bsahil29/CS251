Quicksort is designed as a three-step divide-and-conquer process for sorting an input sequence in an array \cite{cormen}. For any given subarray, $A[i..j]$, the process is as follows,\\
\textbf{Divide}: The array $A[i..j]$ is partitioned into two subarrays $A[i..k]$ and $A[k + 1..j]$ such that all elements in $A[i..k]$ is less than or equal to all elements in $A[k + 1..j]$. A partitioning procedure is called to determine $k$ such that at the end of partitioning, the element at the $k^{th}$ position (i.e., $A[k]$) does not change its position in the final output array.
\begin{algorithm}
  \caption{Partition procedure of \textbf{Quicksort} algorithm.}
  \label{algo:ins_sort1}
  \begin{algorithmic}[1]
     \Procedure{Partition}{$A,i,j$}\newline
     \Comment{A is an array of N integers, $A[1..N]$}\newline
     \Comment{$i$ and $j$ are the start and end of subarray}
      \State $x \leftarrow$ $A[i]$
      \State $y \leftarrow$ $i-1$
      \State $z \leftarrow$ $j+1$
      \While {($true$)}
          \State $z \leftarrow z-1$
           %\Comment{Iterate to find the place for this element in the sorted array}
          \While {($A[z] > x$)}
            \State $z \leftarrow z-1$
          \EndWhile
          \State $y \leftarrow y+1$
          \While {($A[y] < x$)}
            \State $y \leftarrow y+1$ 
          \EndWhile
          \If{$y < z$}
            \State Swap $A[y] \leftrightarrow A[z]$
            \Else
            \State return $x$
          \EndIf
      \EndWhile
     \EndProcedure 
  \end{algorithmic}
\end{algorithm}
\textbf{Conquer}:Recursively invoke \textbf{Quicksort} on the two subarrays. This procedure conquers the complexity by applying the same operations in two subarrays.\\
\textbf{Merge}: Quicksort does not require merge or combine operation as the entire array $A[i..j]$ is sorted in place.\\
\indent In the heart of \textbf{Quicksort}, there is a partition procedure as shown in Algorithm $1$. A pivot element $x$ is selected. The first inner while loop (line \#$6$) continues examining elements until it finds an element that is smaller than or equal to the pivot element. Similarly, the second inner while loop (line \#$9$) continues examining elements until it finds an element that is greater than or equal to the pivot element. If indices $y$ and $z$ have not exchanged their side around the pivot, the elements at $A[y]$ and $A[z]$ are exchanged. Otherwise, the procedure returns the index $z$, such that all elements to the left of $z$ are smaller than or equal to $A[z]$ and all elements to the right of $z$ are greater than or equal to $A[z]$.\\
\indent The main recursive procedure for \textbf{Quicksort} is presented in Algorithm $2$. Initial invocation is performed by call QUICKSORT$(A,1,N)$ where $N$ is the length of array $N$.
\begin{algorithm}
  \caption{\textbf{Quicksort} recursion.}
  \label{algo:ins_sort1}
  \begin{algorithmic}[1]
     \Procedure{Quicksort}{$A,i,j$}\newline
     \Comment{Quicksort procedure called with $A, 1, N$}
          \If{$i < j$}
            \State $k \leftarrow$ PARTITION$(A,i,j)$
            \State QUICKSORT$(A,i,k)$
            \State QUICKSORT$(A,k+1,j)$
          \EndIf
     \EndProcedure 
  \end{algorithmic}
\end{algorithm}
\subsection{Time complexity analysis of
Quicksort}
The worst case of \textbf{Quicksort} occurs when an array of length $N$ , gets partitioned into two subarrays of size $N-1$ and $1$ in every recursive invocation of QUICKSORT procedure in Algorithm $2$. The partitioning procedure presented in Algorithm $1$, takes $\Theta(n)$ time, the recurrence relation for running time is,
%\begin{equation*}
    $$T(n)=T(n-1)+\Theta(n)$$
%\end{equation*}
As it is evident that $T(1) = \Theta(1)$, the recurrence is
solved as follows,
%\begin{equation*}
    $$T(n)=T(n-1)+\Theta(n)$$
    $$=\sum_{k=1}^{n} \Theta(k)$$
%\end{equation*}
 $$=\Theta\Bigg(\sum_{k=1}^{n} k\Bigg)$$
 $$=\Theta(n^2)$$
\indent Therefore, if the partitioning is always maximally unbalanced, the running time is $\Theta(n^2)$. Intutively, if an input sequence is almost sorted, \textbf{Quicksort} will perform poorly. In the best case, partitioning divides the array into two equal parts. Thus, the recurrence for the best case is given by,
$$T(n)=2T\Bigg(\dfrac{n}{2}\Bigg)+\Theta(n)$$
which evaluates to $\Theta(n log_2 n)$. Using a comparatively involved analysis, the average running time of \textbf{Quicksort} can be determined to be $O(n$ lg$ n)$.
