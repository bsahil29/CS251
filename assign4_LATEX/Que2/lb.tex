An interesting question about sorting algorithms based on comparisons is the following: What is the lower bound of this class of sorting algorithms? This question is important for algorithm researchers to further improve the sorting algorithms.\\
\indent A decision tree based analysis leads to the fol- lowing theorem \cite{cormen}.
\newtheorem{theorem}{Theorem}
%\newproof{proof}{Proof}
\begin{theorem}
Any decision tree that sorts $n$ elements has height $\Omega(n log_2(n)$.
\end{theorem}
\begin{proof}
Consider a decision tree of height $h$ that sorts $n$ elements. Since there are $n!$ permutations of $n$ elements, each permutation representing a distinct sorted order, the tree must have at least $n!$ leaves. Since a binary tree of height $h$ has no more than $2^h$ leaves. So,\\
$n! \leq 2^h$\\
Applying logarithmic $(log_2)$, the inequality becomes,\\
$h \geq lg(n!).$\\
Applying Stirling’s approximations,\\
$n! > \Bigg(\dfrac{n}{e}\Bigg)^n$,\\
where e is natural base of logarithms. Further,
$$h \geq lg\Bigg(\dfrac{n}{e}\Bigg)^n$$
$$=n lg n - n lg e$$
$$=\Omega(n lg n)$$
\end{proof}