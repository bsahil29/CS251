Sorting is one of the most fundamental operations in computer science useful in numerous applica- tions. Given a sequence of numbers as input, the output should provide a non-decreasing sequence of numbers as output. More formally, we define a sorting problem as follows \cite{cormen},\\
\textbf{Input}: A sequence of $n$ numbers $<a_1, a_2, ..., a_n>$.\\
\textbf{Output}: A reordered sequence (of size n) $<a_1', a_2', ..., a_n'>$ of the input sequence such that $a_1' \leq a_2' \leq...\leq a_n'$.\\
Consider the following example. Given an input sequence $<8, 34, 7, 9, 15, 91, 15>$, a sorting algorithm should return $<7, 8, 9, 15, 15, 34, 91>$ as output.\\
\indent A fundamental problem like sorting has attracted many researchers who contributed with innovative algorithms to solve the problem of sorting \textit{efficiently} \cite{martin}. Efficiency of an algorithm depends on primarily on two aspects,\\
\begin{itemize}
    \item \textbf{Time complexity} is a formalism that captures running time of an algorithm in terms of the input size. Normally, asymptotic behavior on the input size is used to analyze the time complexity of algorithms.
    \item \textbf{Space complexity} is a formalism that cap- tures amount of memory used by an algorithm in terms of input size. Like time complexity analysis, asymptotic analysis is used for space complexity.
\end{itemize}
In the branch of algorithms and complexity in computer science, space complexity takes a back seat compared to time complexity. Recently, another parameter of computing i.e., energy consumption has become popular. Roy et al. \cite{roy} proposed an energy complexity model for algorithms. In this doc- ument, we will deal with time complexity of sorting algorithms.\\
\indent One class of algorithms which are based on element comparison are commonly known as comparison based sorting algorithms. In this document we will provide a brief overview of \textbf{Quicksort}, a commonly used comparison based sorting algorithm \cite{hoare}. Quicksort is a sorting algorithm based on \textit{divide-and-conquer} paradigm of algorithm design. Further, we will derive the lower bound of any comparison based sorting algorithm to be $\Omega(n log_2 n)$ for an input size of $n$.